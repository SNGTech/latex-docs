\documentclass[12pt]{article}
\usepackage[utf8]{inputenc}
\usepackage{amsmath, amssymb, amsthm}
\usepackage{hyperref}

\usepackage[letterpaper, top=0.8in, bottom=1in, left=1in, right=1in, heightrounded]{geometry}
\usepackage{titling}
\renewcommand\maketitlehooka{\null\mbox{}\vfill}
\renewcommand\maketitlehookd{\vfill\null}

\renewcommand{\baselinestretch}{1.15}

\setlength{\parindent}{0pt}
\setlength{\parskip}{1.8em}

\title{Pre-University Mathematics}
\author{\textbf{Titus Lim}}
\date{June 2025}

\begin{document}

\begin{titlingpage}
	\maketitle
\end{titlingpage}

\tableofcontents
\newpage

\section*{Introduction}

This paper consists of many different pre-university topics primarily from Cambridge A-Level H2 and H3 mathematics such as Fourier, Maclaurin Series, Conics, Equation of Planes etc.
It covers the derivation and applications of the different topics. Some examples are also given.

\section{Sequences and Series}

Consider the following infinite series of elements:
\[
	\frac{1}{2}, \frac{1}{4}, \frac{1}{8},\dots,\frac{1}{2^{i}},\dots
\]

Can we assign this infinite sum to a numerical value? Indeed, we can. We must first understand what are \textbf{series} and \textbf{sequences}.

\subsection{Sequence}

A sequence is any number of elements arranged in a specific order.

\textbf{Example 1.} An infinite sequence of ascending odd numbers.
\[
	1, 3, 5, 7, \dots
\]
Sequences can be both infinite and finite. An example of a finite sequence is:
\[
	2, 4, 8, 16, \dots, n
\]
where $n$ is the final element in the sequence.

More formally, the algebraic notation for a sequence is expressed as:
\[
	a_1, a_2, a_3, \dots, a_n
\]
where $a_1$ is the first term, $a_2$ is the second term, and $a_n$ is the $n^{th}$ term.

\subsection{Series}

A series is the total sum of all elements in a sequence. In the first section, we posed the question of obtaining a numerical value from an infinite sum. What we are really asking is: \textit{How do we evaluate a series?}

With the initial sequence in \textbf{Section 1}, let's first change it to a finite sequence. We can rewrite it into this series:
\[
	S_{n} = \frac{1}{2} + \frac{1}{4} + \frac{1}{6} + \dots + \frac{1}{2^{n}}
\]

We can also write the above in summation notation:
\[
	S_{n} = \displaystyle\sum_{k=1}^{n}\frac{1}{2^{k}}
\]
This describes the sum of elements of $\displaystyle\frac{1}{2^{n}}$ where $k=1$ to $k=n$.

Suppose we want to find the sum of 10 elements in the series. We can write it as:
\begin{align*}
	S_{10} & = \displaystyle\sum_{k=1}^{10}\frac{1}{2^{k}}                                    \\
	       & = \frac{1}{2^{1}} + \frac{1}{2^{2}} + \frac{1}{2^{3}} + \dots + \frac{1}{2^{10}} \\
         & = 0.99902
\end{align*}
In the next sections, we will explore the different types of series.

\subsection{Arithmetic Progression}



\subsection{Geometric Progression}

\end{document}
