\documentclass[12pt]{article}
\usepackage[utf8]{inputenc}
\usepackage{amsmath, amssymb, amsthm}
\usepackage{hyperref}

\usepackage[letterpaper, top=0.8in, bottom=1in, left=1in, right=1in, heightrounded]{geometry}
\usepackage{titling}
\renewcommand\maketitlehooka{\null\mbox{}\vfill}
\renewcommand\maketitlehookd{\vfill\null}

\renewcommand{\baselinestretch}{1.15}

\setlength{\parindent}{0pt}
\setlength{\parskip}{1.8em}

\title{Pre-University Mathematics}
\author{\textbf{Titus Lim}}
\date{June 2025}

\begin{document}

\begin{titlingpage}
	\maketitle
\end{titlingpage}

\tableofcontents
\newpage

\section*{Introduction}

This paper consists of many different pre-university topics primarily from Cambridge A-Level H2 and H3 mathematics such as Fourier, Maclaurin Series, Conics, Equation of Planes etc.
It covers the derivation and applications of the different topics. Some examples are also given.

\section{Sequences and Series}

Consider the following infinite series of elements:
\[
	\frac{1}{2}, \frac{1}{4}, \frac{1}{8},\dots,\frac{1}{2^{i}},\dots
\]

Can we assign this infinite sum to a numerical value? Indeed, we can. We must first understand what are \textbf{series} and \textbf{sequences}.

\subsection{Sequence}

A sequence is any number of elements arranged in a specific order.

\textbf{Example 1.} An infinite sequence of ascending odd numbers.
\[
	1, 3, 5, 7, \dots
\]
Sequences can be both infinite and finite. An example of a finite sequence is:
\[
	2, 4, 8, 16, \dots, n
\]
where $n$ is the final element in the sequence.

More formally, the algebraic notation for a sequence is expressed as:
\[
	a_1, a_2, a_3, \dots, a_n
\]
where $a_1$ is the first term, $a_2$ is the second term, and $a_n$ is the $n$-th term.

\subsection{Series}

A series is the total sum of all elements in a sequence. In the first section, we posed the question of obtaining a numerical value from an infinite sum. What we are really asking is: \textit{How do we evaluate a series?}

With the initial sequence in \textbf{Section 1}, let's first change it to a finite sequence. We can rewrite it into this series:
\[
	S_{n} = \frac{1}{2} + \frac{1}{4} + \frac{1}{6} + \dots + \frac{1}{2^{n}}
\]

We can also write the above in summation notation:
\[
	S_{n} = \displaystyle\sum_{k=1}^{n}\frac{1}{2^{k}}
\]
This describes the sum of elements of $\displaystyle\frac{1}{2^{n}}$ where $k=1$ to $k=n$.

Suppose we want to find the sum of 10 elements in the series. We can write it as:
\begin{align*}
	S_{10} & = \displaystyle\sum_{k=1}^{10}\frac{1}{2^{k}}                                    \\
	       & = \frac{1}{2^{1}} + \frac{1}{2^{2}} + \frac{1}{2^{3}} + \dots + \frac{1}{2^{10}} \\
	       & = 0.99902
\end{align*}
In the next sections, we will explore the different types of sequences and series.

\subsection{Arithmetic Progression}

The first type of sequence is known as an \textbf{arithmetic progression}.

\textbf{Example 2.} Consider the following sequence:
\[
	2, 5, 8, 11, \dots
\]
To obtain the above sequence, we start with the first term and add a fixed value to each term successively.
\[
	2, 2 + 3 = 5, 5 + 3 = 8, 8 + 3 = 11, \dots
\]
This fixed value is called the \textbf{common difference}. In the above example, the common difference would be 3.

More formally, an \textbf{arithmetic progression} or \textbf{AP} is a sequence whereby the difference between the preceding and succeeding terms is common.
In algebraic notation, it can be written as:
\[
	a, a + d, a + 2d, a + 3d, \dots, a + (n-1)d
\]
where $a$ is the first term, $d$ is the common difference, and $n$ is the number of terms in the sequence.

From the above, we can observe that the $n$-th term is:
\[
	a_n = a + (n-1)d
\]

\subsubsection{Sum of AP Series}

An AP series is simply the total sum of terms in a given arithmetic progression. For a general arithmetic progression:
\[
	a, (a + d), (a + 2d), \dots, (\ell - 2d), (\ell - d), \ell
\]
where $a$ is the first term, $d$ is the common difference and $\ell$ is the $n$-th term.

The AP series is written as:
\[
	S_n = a + (a + d) + (a + 2d) + \dots + (\ell - 2d) + (\ell - d) + \ell
\]
To derive a general formula for the AP series, let's first reverse the order of $S_n$:
\[
	S_n = \ell + (\ell - d) + (\ell - 2d) + \dots + (a + 2d) + (a + d) + a
\]
Then, let's add the original AP series to the reverse AP series by vertically summing each term.
\[
	2S_n = (a + \ell) + (a + \ell) + (a + \ell) + \dots + (a + \ell) + (a + \ell) + (a + \ell)
\]
Notice how all the terms have become $(a + \ell)$. There are now $n$ number of $(a + \ell)$ terms. Now, we can make $S_n$ the subject:
\[
	S_n = \frac{1}{2}n(a + \ell)
\]
As seen in the previous section, the $n-th$ term, $\ell$, can also be written as $a + (n - 1)d$.
\textbf{Derivation.} Hence, the general formula for the sum of any AP series is:
\[
	S_n = \frac{1}{2}n(2a + (n-1)d)
\]

\subsection{Geometric Progression}

The second type of sequence is known as a \textbf{geometric progression}.

\textbf{Example 3.} Consider the following sequence:
\[
	2, 6, 18, 54, \dots
\]
To obtain the above sequence, we start with the first term and multiply it by a fixed value to each term successively.
\[
	2, 2 \times 3 = 6, 6 \times 3 = 18, 18 \times 3 = 54, \dots
\]
This fixed value is called the \textbf{common ratio}. In the above example, the common ratio would be 3.

More formally, a \textbf{geometric progression} or \textbf{GP} is a sequence whereby each term is produced by multiplying each preceding term by a constant value. In algebraic notation, it can be written as:
\[
  a, ar, ar^2, ar^3, \dots, ar^{n-1}
\]
where $a$ is the first term, $r$ is the common ratio, and $n$ is the number of terms in the sequence.

From the above, we can observe that the $n$-th term is:
\[
  a_n = ar^{n-1}
\]

\subsubsection{Sum of GP Series}

Similar to a series, a GP Series is simply the total sum of terms in a given geometric progression. For a general geometric progression, its GP series can be written as:
\[
  S_n = a + ar + ar^2 + ar^3 + \dots + ar^{n-1}
\]


\end{document}
