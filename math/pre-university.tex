\documentclass[12pt]{article}
\usepackage[utf8]{inputenc}
\usepackage{amsmath, amssymb, amsthm}
\usepackage{hyperref}

\usepackage[letterpaper, top=0.8in, bottom=1in, left=1in, right=1in, heightrounded]{geometry}
\usepackage{titling}
\renewcommand\maketitlehooka{\null\mbox{}\vfill}
\renewcommand\maketitlehookd{\vfill\null}

\renewcommand{\baselinestretch}{1.15}

\setlength{\parindent}{0pt}
\setlength{\parskip}{1.8em}

\title{Pre-University Mathematics}
\author{\textbf{Titus Lim}}
\date{June 2025}

\begin{document}

\begin{titlingpage}
	\maketitle
\end{titlingpage}

\tableofcontents
\newpage

\section*{Introduction}

This paper consists of many different pre-university topics primarily from Cambridge A-Level H2 and H3 mathematics such as Fourier, Maclaurin Series, Conics, Equation of Planes etc.
It covers the derivation and applications of the different topics. Some examples are also given.

\section{Sequences and Series}

Consider the following infinite series of elements:
\[
	\frac{1}{2}, \frac{1}{4}, \frac{1}{8},\dots,\frac{1}{2^{i}},\dots
\]

Can we assign this infinite sum to a numerical value? Indeed, we can. We must first understand what are \textbf{series} and \textbf{sequences}.

\subsection{Sequence}

A sequence is any number of elements arranged in a specific order.

\textbf{Example 1.} An infinite sequence of ascending odd numbers.
\[
	1, 3, 5, 7, \dots
\]
Sequences can be both infinite and finite. An example of a finite sequence is:
\[
	2, 4, 8, 16, \dots, n
\]
where $n$ is the final element in the sequence.

More formally, the algebraic notation for a sequence is expressed as:
\[
	a_1, a_2, a_3, \dots, a_n
\]
where $a_1$ is the first term, $a_2$ is the second term, and $a_n$ is the $n$-th term.

\subsection{Series}

A series is the total sum of all elements in a sequence. In the first section, we posed the question of obtaining a numerical value from an infinite sum. What we are really asking is: \textit{How do we evaluate a series?}

With the initial sequence in \textbf{Section 1}, let's first change it to a finite sequence. We can rewrite it into this series:
\[
	S_{n} = \frac{1}{2} + \frac{1}{4} + \frac{1}{6} + \dots + \frac{1}{2^{n}}
\]

We can also write the above in summation notation:
\[
	S_{n} = \displaystyle\sum_{k=1}^{n}\frac{1}{2^{k}}
\]
This describes the sum of elements of $\displaystyle\frac{1}{2^{n}}$ where $k=1$ to $k=n$.

Suppose we want to find the sum of 10 elements in the series. We can write it as:
\begin{align*}
	S_{10} & = \displaystyle\sum_{k=1}^{10}\frac{1}{2^{k}}                                    \\
	       & = \frac{1}{2^{1}} + \frac{1}{2^{2}} + \frac{1}{2^{3}} + \dots + \frac{1}{2^{10}} \\
	       & = 0.99902
\end{align*}

\subsection{Arithmetic Progression}

The first type of sequence is known as an \textbf{arithmetic progression}.

\textbf{Example 2.} Consider the following sequence:
\[
	2, 5, 8, 11, \dots
\]
To obtain the above sequence, we start with the first term and add a fixed value to each term successively.
\[
	2, 2 + 3 = 5, 5 + 3 = 8, 8 + 3 = 11, \dots
\]
This fixed value is called the \textbf{common difference}. In the above example, the common difference would be 3.

More formally, an \textbf{arithmetic progression} or \textbf{AP} is a sequence whereby the difference between the preceding and succeeding terms is common.
In algebraic notation, it can be written as:
\[
	a, a + d, a + 2d, a + 3d, \dots, a + (n-1)d
\]
where $a$ is the first term, $d$ is the common difference, and $n$ is the number of terms in the sequence.

From the above, we can observe that the $n$-th term is:
\[
	a_n = a + (n-1)d
\]

\subsubsection{Sum of AP Series}

An AP series is simply the total sum of terms in a given arithmetic progression. For a general arithmetic progression:
\[
	a, (a + d), (a + 2d), \dots, (\ell - 2d), (\ell - d), \ell
\]
where $a$ is the first term, $d$ is the common difference and $\ell$ is the $n$-th term.

The AP series is written as:
\[
	S_n = a + (a + d) + (a + 2d) + \dots + (\ell - 2d) + (\ell - d) + \ell
\]
To derive a general formula for the AP series, let's first reverse the order of $S_n$:
\[
	S_n = \ell + (\ell - d) + (\ell - 2d) + \dots + (a + 2d) + (a + d) + a
\]
Then, let's add the original AP series to the reverse AP series by vertically summing each term.
\[
	2S_n = (a + \ell) + (a + \ell) + (a + \ell) + \dots + (a + \ell) + (a + \ell) + (a + \ell)
\]
Notice how all the terms have become $(a + \ell)$. There are now $n$ number of $(a + \ell)$ terms. Now, we can make $S_n$ the subject:
\[
	S_n = \frac{1}{2}n(a + \ell)
\]
As seen in the previous section, the $n-th$ term, $\ell$, can also be written as $a + (n - 1)d$.  \\
\textbf{Derivation.} Hence, the general formula for the sum of any AP series is:
\[
	S_n = \frac{1}{2}n(2a + (n-1)d)
\]
All AP series with common difference, $d \neq 0$ are \textbf{divergent}. For any AP series,  \\
If $d > 0$, then the sum of any AP series diverges to $+\infty$ as $n \rightarrow \infty$.  \\
If $d < 0$, then the sum of any AP series diverges to $-\infty$ as $n \rightarrow \infty$.  \\
If $d = 0$ and $a \neq 0$, then the sum of any AP series diverges to $+-\infty$ as $n \rightarrow$ depending on the sign of $a$. \\
If $d = 0$ and $a = 0$, then the sum of any AP series is $0$.

\subsection{Geometric Progression}

The second type of sequence is known as a \textbf{geometric progression}.

\textbf{Example 3.} Consider the following sequence:
\[
	2, 6, 18, 54, \dots
\]
To obtain the above sequence, we start with the first term and multiply it by a fixed value to each term successively.
\[
	2, 2 \times 3 = 6, 6 \times 3 = 18, 18 \times 3 = 54, \dots
\]
This fixed value is called the \textbf{common ratio}. In the above example, the common ratio would be 3.

More formally, a \textbf{geometric progression} or \textbf{GP} is a sequence whereby each term is produced by multiplying each preceding term by a constant value. In algebraic notation, it can be written as:
\[
	a, ar, ar^2, ar^3, \dots, ar^{n-1}
\]
where $a$ is the first term, $r$ is the common ratio, and $n$ is the number of terms in the sequence.

From the above, we can observe that the $n$-th term is:
\[
	a_n = ar^{n-1}
\]

\subsubsection{Sum of GP Series}

Similar to a series, a GP Series is simply the total sum of terms in a given geometric progression. For a general geometric progression, its GP series can be written as:
\[
	S_n = a + ar + ar^2 + ar^3 + \dots + ar^{n-1}
\]
We can evaluate the series by multiplying both sides by $r$ and then subtracting:
\begin{align*}
	rS_n       & = ar + ar^2 + ar^3 + ar^4 + \dots + ar^{n-1} + ar^n \\
	S_n - rS_n & = a - ar^n                                          \\
	S_n(1 - r) & = a - ar^n                                          \\
\end{align*}
Now, we can make $S_n$ the subject.\\
\textbf{Derivation.} Hence, the general formula for the sum of any GP series is:
\[
	S_n = \frac{a - ar^n}{1 - r}   \;\;\;\;   \text{for } r \neq 1
\]
A GP series can either be \textbf{convergent} or \textbf{divergent}.

\subsection{p-Series}

The \textbf{p-Series} is a type of series in the form:
\[
	\displaystyle\sum_{n=1}^{\infty}\frac{1}{n^p} = 1 + \frac{1}{2^p} + \frac{1}{3^p} + \dots + \frac{1}{n^p} + \dots
\]
where $p$ is an exponent and $p \geq 0$.

Interestingly, when $p = 1$, it forms a special type of series called a \textbf{Harmonic Series}:
\[
	\displaystyle\sum_{n=1}^{\infty}\frac{1}{n} = 1 + \frac{1}{2} + \frac{1}{3} + \dots + \frac{1}{n} + \dots
\]

In general, a p-series converges if $p > 1$ as the partial sum grows smaller fast enough. Conversely, it diverges if $p \leq 1$.

\subsection{Convergence and Divergence of Series}
A \textbf{convergent} series is any series where the sum converges to a numerical value. For a given series
\[
	\displaystyle\sum_{n=1}^{\infty}a_n
\]
we say it is convergent if the series converges to the sum, $S \in \mathbb{R}$, if
\[
	\displaystyle\lim_{n\rightarrow\infty}S_n = S
\]
where the partial sum
\[
	S_n = a_1 + a_2 + a_3 + \dots + a_n = \displaystyle\sum_{k=1}^{n}a_k
\]
Otherwise, the series is \textbf{divergent}.

Distributive and associative arithmetic laws still apply for any series.

\subsubsection{Absolute Convergence}

The series
\[
	\displaystyle\sum_{n=1}^{\infty}a_n
\]
converges \textbf{absolutely} if
\[
	\displaystyle\sum_{n=1}^{\infty}|a_n| \;\; \text{converges}
\]
Otherwise, it converges \textbf{conditionally} if
\[
	\displaystyle\sum_{n=1}^{\infty}a_n \;\; \text{converges, but } \displaystyle\sum_{n=1}^{\infty}|a_n| \;\; \text{diverges}
\]

In the next few sections, we will explore how we can test if a given series is convergent or divergent.

\subsubsection{Divergence Test}

The divergence test states that a series is divergent if the limit of the term $a_n \neq 0$ or does not exist.

\textbf{Definition.} The series diverges if
\begin{gather*}
	\displaystyle\lim_{n\rightarrow\infty}a_n \neq 0  \;\;  \text{or}  \\
	\displaystyle\lim_{n\rightarrow\infty}a_n \;\; \text{does not exist}
\end{gather*}
If the limit is $0$, more tests would need to be carried out to determine if the series is truly convergent.

\textbf{Example 4.1} Determine if the following series $\displaystyle\sum_{n=1}^{\infty}\frac{n - 1}{n}$ is convergent or divergent.
\begin{align*}
	  & \displaystyle\lim_{n\rightarrow\infty}\frac{n - 1}{n}   \\
	= & \displaystyle\lim_{n\rightarrow\infty}(1 - \frac{1}{n}) \\
	= & \displaystyle\lim_{n\rightarrow\infty}(1 - 0)           \\
	= & 1
\end{align*}
\[
	\therefore \text{Since the limit $\neq 0$, the series is divergent}.
\]

\subsubsection{Geometric Series Test}

The geometric series test applies to any geometric series in the form $\displaystyle\sum_{n=1}^{\infty}ar^{n-1}$.

\textbf{Definition.} The series converges or diverges based on the following
\[
	\displaystyle\sum_{n=1}^{\infty}ar^{n-1} =
	\begin{cases}
		\displaystyle\frac{a}{1-r} & \text{for } |r| < 1    \\
		\text{diverges}            & \text{for } |r| \geq 1
	\end{cases}
\]

\textbf{Example 4.2} Determine if the following series $\displaystyle\sum_{n=1}^{\infty}\frac{5^{n-1}}{3^{n+1}}$ is convergent or divergent. If it converges, find the value it converges to.
\begin{align*}
	  & \displaystyle\sum_{n=1}^{\infty}\frac{3^{n-1}}{5^{n+1}}                   \\
	= & \displaystyle\sum_{n=1}^{\infty}\frac{1}{25}\left(\frac{3}{5}\right)^{n-1} \\
	= & \frac{\frac{1}{25}}{1-\frac{3}{5}}                                         \\
	= & \frac{1}{10}
\end{align*}

\subsubsection{Comparison Test}



\end{document}
